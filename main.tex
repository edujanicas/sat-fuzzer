\documentclass[11pt,a4paper,titlepage]{article}
\usepackage[utf8]{inputenc}
\usepackage[margin=2cm,headheight=13.6cm]{geometry}
\usepackage{color}
\usepackage{listings}
\usepackage{graphicx}
\usepackage{caption}
\usepackage{float}
\usepackage{titling}
\usepackage{pgfkeys}
\usepackage{listings}
\usepackage{multicol}
\usepackage{helvet}
\definecolor{dkgreen}{rgb}{0,0.6,0}
\definecolor{gray}{rgb}{0.5,0.5,0.5}
\definecolor{mauve}{rgb}{0.58,0,0.82}
\definecolor{mygreen}{RGB}{0,127,0}
\definecolor{mygray}{RGB}{100,100,100}
\definecolor{mymauve}{RGB}{100,32,255}
\definecolor{lgray}{RGB}{230,230,230}
\lstset{frame=tb,
  language=C,
  aboveskip=3mm,
  belowskip=3mm,
  showstringspaces=false,
  columns=flexible,
  basicstyle={\small\ttfamily},
  numbers=none,
  numberstyle=\tiny\color{gray},
  keywordstyle=\color{blue},
  commentstyle=\color{dkgreen},
  stringstyle=\color{mauve},
  breaklines=true,
  breakatwhitespace=true,
  tabsize=3
}
\lstset{ %
  frame=none,
  backgroundcolor=\color{white},   % choose the background color; you must add \usepackage{color} or \usepackage{xcolor}
  basicstyle=\footnotesize\ttfamily,        % the size of the fonts that are used for the code
  breakatwhitespace=false,         % sets if automatic breaks should only happen at whitespace
  breaklines=true,                 % sets automatic line breaking
  captionpos=t,                    % sets the caption-position to bottom
  commentstyle=\color{mygreen},    % comment style
  deletekeywords={...},            % if you want to delete keywords from the given language
  escapeinside={\%*}{*)},          % if you want to add LaTeX within your code
  extendedchars=true,              % lets you use non-ASCII characters; for 8-bits encodings only, does not work with UTF-8
%  frame=single,                    % adds a frame around the code
  keepspaces=true,                 % keeps spaces in text, useful for keeping indentation of code (possibly needs columns=flexible)
  keywordstyle=\color{blue},       % keyword style
  language=,                 % the language of the code
  morekeywords={*,...},            % if you want to add more keywords to the set
  numbers=left,                    % where to put the line-numbers; possible values are (none, left, right)
  numbersep=5pt,                   % how far the line-numbers are from the code
  numberstyle=\tiny\color{mygray}, % the style that is used for the line-numbers
  rulecolor=\color{black},         % if not set, the frame-color may be changed on line-breaks within not-black text (e.g. comments (green here))
  showspaces=false,                % show spaces everywhere adding particular underscores; it overrides 'showstringspaces'
  showstringspaces=false,          % underline spaces within strings only
  showtabs=false,                  % show tabs within strings adding particular underscores
  stepnumber=1,                    % the step between two line-numbers. If it's 1, each line will be numbered
  stringstyle=\color{mymauve},     % string literal style
  tabsize=4,                       % sets default tabsize to 2 spaces
  aboveskip=3mm,
  belowskip=3mm,
}

\usepackage{fancyhdr}
\pagestyle{fancy}
\rhead{Eduardo Janicas \& Luca Battistelli \& Luca Campese}
\usepackage{datetime}
\usepackage{moresize}

\newdateformat{monthyeardate}{%
  \monthname[\THEMONTH] \THEYEAR}

\newcommand{\rulebreak}{%
	\par%
	\vspace{0.9cm}%
    \noindent\rule{4cm}{0.4pt}%
    \vspace{1.2cm}%
    \par%
}

\newcommand{\coverpage}[1]{%
	\pagenumbering{roman}%
	\thispagestyle{empty}%
	\lhead{\textsc{440 Software Reliability – #1}}%
    \title{440 Software Reliability}%
    \author{Eduardo Janicas \and Luca Battistelli \and Luca Campese}%
    \newgeometry{left=5cm,bottom=2cm,right=5cm,top=2cm}%
	\begin{center}\hspace{0pt}\vfill%
    \uppercase{Department of Computing\\
    Imperial College of London}
	\rulebreak%
    {\Large\textbf{440 Software Reliability}}
    
    \vspace{0.5cm}
    {\HUGE\textbf{\textit{#1}}}
    
    \vspace{0.5cm}
	\theauthor%
	\par%
	\vspace{0.9cm}%
    \noindent\rule{4cm}{0.4pt}%
    \vspace{0.45cm}
    \tableofcontents%
	\rulebreak%
    \monthyeardate\today\par
    \hspace{0pt}
	\end{center}%
    \vfill
    \hspace{0pt}
	\pagebreak%
    \restoregeometry%
    \pagenumbering{arabic}%
}

% Custom arguments for /fig command
\pgfkeys{
 /fig/.is family, /fig,
 default/.style = 
  {scale = 1,
   angle = 0},
 scale/.estore in = \figScale,
 angle/.estore in = \figAngle
}
\newcommand{\fig}[2][]{%
	\pgfkeys{/fig, default, #1}%
	\begin{figure}[H]%
    \centering
    \includegraphics[angle=\figAngle,width=\figScale\textwidth]{#2}%
	\end{figure}%
}

\newcommand{\filename}[1]{%
	\texttt{#1}%
}

\newcommand{\vhdl}[1]{%
  \lstinputlisting[language=vhdl]{#1}
}

\newcommand*\paths[1]{\lstset{inputpath=#1}\graphicspath{#1}}



\begin{document}
\coverpage{Fuzzer}
  
\section{Introduction}  
We have implemented a black-box fuzzer using the Python programming language. It can be run in two different modes, which will be referred to as \texttt{ub} mode and \texttt{func} mode in the following sections. It also makes use of a seed provided as an argument which allows our fuzzer to exhibit a deterministic behaviour.

\section{\texttt{ub} mode}


\subsection(Evaluating \texttt{func} mode)


\section{\texttt{func} mode}
When using this mode, the fuzzer behaves in a \textit{blind} fashion, generating new inputs by mutating pre-existing formulae. Only valid inputs will be generated in this phase.

We have developed a set of rules to create new formulae, which can be chained together to generate more complex transformations of existing inputs.

These rules are outlined below:

\begin{itemize}  
\item 0 - 4 Swap clauses (1)
\item 5 - 9 Swap literals inside a clause (2)
\item 10 - 14 Add clauses (3)
\item 15 - 19 Remove Clauses (4)
\item 20 - 24 (1+2)
\item 25 - 29 (1+3)
\item 30 - 34 (1+4)
\item 35 - 39 (2+3)
\item 40 - 44 (1+2+3)
\item 45 - 49 (1+2+4)
\end{itemize}

The first rule, (1), swaps clauses of a formula in a semi-random order. This guarantees the result is the same, but it may cause, in some solvers, to take additional time 
and to exercise different functions of the solver.´

The second rule, (2), keeps the order of the formula clauses, but randomly swaps the literals inside each clause. It also know that the last value of any line (clause) in the 
DIMACS format is 0, so it keeps that value in the end. This transformation also keeps the satisfiability of the formula unchanged, but may cause propagation to behave diferently.

The third rule, (3), adds a random number of clases to the formula. We keep this number between 0 and N, where N is the total number of clauses that already existed in the formula.
This prevents a test case from growing to a much higher difficulty that the solvers can't handle. For the same reason, each clause we add is limited to 10 literals, that obviously 
conform to the literal identifiers we already had. 
This technique mantains the satisfiability of an UNSAT clause, i.e., if there was a conflict, that conflict still exists, but now we cannot infer anything about being, or not, SAT.
The added clauses may create a new conflict that make the original SAT formula now UNSAT.

In the fourth rule (4), the opposite happens. From a formula, we remove between 0 and N clauses, where N is the total number of clauses of the formula.
If the formula was SAT, removing clauses will not introduce conflicts, and therefore it will remain SAT. On the other hand, if the formula was UNSAT, we might have
removed the clauses that caused the conflict, making the new formula SAT. We cannot assume satisfiability in this case.

The next rules consist on different combinations of the latter:

We can combine (1) and (2), which will not change the satisfiabiliy of the formula because it just shuffles the clauses inside and the literals inside a clause.

If we combine (1) and (3), although (1) doesn't change satisfiability, adding clauses after the shuffling causes the same satisfiability problem seen in (3).
If we combine (1) and (4), although (1) doesn't change satisfiability, adding clauses after the shuffling causes the same satisfiability problem seen in (4).
If we combine (2) and (3), although (2) doesn't change satisfiability, adding clauses after the shuffling causes the same satisfiability problem seen in (3).

In the end, we can either combine (1), (2) and (3), which has the same result in satisfiability as just (3), or we can combine (1), (2) and (4). 
Notice that we cannot ever combine (3) and (4), because adding random clauses and then deleting random clauses would leave us with a formula we didn't know nothing about.

It is also worth noting that when implementing these transformations, we had to be carefull with the DIMACS representation. For instance, when adding/deleting clauses 
we had to keep track of the ammount of new/removed clauses to update the first line of the file.

\subsection(Evaluating \texttt{func} mode)
We began by only applying transformations 1 and 2 to the SUTS. By running the script sat-follow-up, we begin by observing coverage of around 30\%.
Incrementaly, we added the transformations described above and checked the results with sat-follow-up, going up to arround 90\% of code coverage.
To get this result, we also needed to add a timeout to this script, because some of the larger inputs started causing infinite loops on sut1 and sut3.

\section(Next Steps)


\end{document}